\subsection{keyproc}
\begin{lstlisting}[language=C]
/****************************************************************************/
/*                                                                          */
/*                                KEYPROC.H                                 */
/*                         Key Processing Functions                         */
/*                               Include File                               */
/*                       Digital Oscilloscope Project                       */
/*                                 EE/CS 52                                 */
/*                                                                          */
/****************************************************************************/

/*
   This file contains the constants and function prototypes for the key
   processing functions (defined in keyproc.c) for the Digital Oscilloscope
   project.


   Revision History:
      3/8/94   Glen George       Initial revision.
      3/13/94  Glen George       Updated comments.
*/



#ifndef  __KEYPROC_H__
    #define  __KEYPROC_H__


/* library include files */
  /* none */

/* local include files */
#include  "scopedef.h"




/* constants */
    /* none */




/* structures, unions, and typedefs */
    /* none */




/* function declarations */

enum status  no_action(enum status);      /* nothing to do */

enum status  menu_key(enum status);	  /* process the <Menu> key */

enum status  menu_up(enum status);	  /* <Up> key in a menu */
enum status  menu_down(enum status);	  /* <Down> key in a menu */
enum status  menu_left(enum status);	  /* <Left> key in a menu */
enum status  menu_right(enum status);	  /* <Right> key in a menu */


#endif
\end{lstlisting}

\begin{lstlisting}[language=C]
/****************************************************************************/
/*                                                                          */
/*                                 KEYPROC                                  */
/*                         Key Processing Functions                         */
/*                       Digital Oscilloscope Project                       */
/*                                 EE/CS 52                                 */
/*                                                                          */
/****************************************************************************/

/*
   This file contains the key processing functions for the Digital
   Oscilloscope project.  These functions are called by the main loop of the
   system.  The functions included are:
      menu_down  - process the <Down> key while in a menu
      menu_key   - process the <Menu> key
      menu_left  - process the <Left> key while in a menu
      menu_right - process the <Right> key while in a menu
      menu_up    - process the <Up> key while in a menu
      no_action  - nothing to do

   The local functions included are:
      none

   The locally global variable definitions included are:
      none


   Revision History
      3/8/94   Glen George       Initial revision.
      3/13/94  Glen George       Updated comments.
*/



/* library include files */
  /* none */

/* local include files */
#include  "scopedef.h"
#include  "keyproc.h"
#include  "menu.h"




/*
   no_action

   Description:      This function handles a key when there is nothing to be
                     done.  It just returns.

   Arguments:        cur_state (enum status) - the current system state.
   Return Value:     (enum status) - the new system state (same as current
   		     state).

   Input:            None.
   Output:           None.

   Error Handling:   None.

   Algorithms:       None.
   Data Structures:  None.

   Global Variables: None.

   Author:           Glen George
   Last Modified:    Mar. 8, 1994

*/

enum status  no_action(enum status cur_state)
{
    /* variables */
      /* none */



    /* return the current state */
    return  cur_state;

}




/*
   menu_key

   Description:      This function handles the <Menu> key.  If the passed
                     state is MENU_ON, the menu is turned off.  If the passed
		     state is MENU_OFF, the menu is turned on.  The returned
		     state is the "opposite" of the passed state.

   Arguments:        cur_state (enum status) - the current system state.
   Return Value:     (enum status) - the new system state ("opposite" of the
   		     as current state).

   Input:            None.
   Output:           The menu is either turned on or off.

   Error Handling:   None.

   Algorithms:       None.
   Data Structures:  None.

   Global Variables: None.

   Author:           Glen George
   Last Modified:    Mar. 8, 1994

*/

enum status  menu_key(enum status cur_state)
{
    /* variables */
      /* none */



    /* check if need to turn the menu on or off */
    if (cur_state == MENU_ON)
        /* currently the menu is on, turn it off */
	clear_menu();
    else
        /* currently the menu is off, turn it on */
	display_menu();


    /* all done, return the "opposite" of the current state */
    if (cur_state == MENU_ON)
        /* state was MENU_ON, change it to MENU_OFF */
        return  MENU_OFF;
    else
        /* state was MENU_OFF, change it to MENU_ON */
        return  MENU_ON;

}




/*
   menu_up

   Description:      This function handles the <Up> key when in a menu.  It
                     goes to the previous menu entry and leaves the system
		     state unchanged.

   Arguments:        cur_state (enum status) - the current system state.
   Return Value:     (enum status) - the new system state (same as current
   		     state).

   Input:            None.
   Output:           The menu display is updated.

   Error Handling:   None.

   Algorithms:       None.
   Data Structures:  None.

   Global Variables: None.

   Author:           Glen George
   Last Modified:    Mar. 8, 1994

*/

enum status  menu_up(enum status cur_state)
{
    /* variables */
      /* none */



    /* go to the previous menu entry */
    previous_entry();


    /* return the current state */
    return  cur_state;

}




/*
   menu_down

   Description:      This function handles the <Down> key when in a menu.  It
                     goes to the next menu entry and leaves the system state
		     unchanged.

   Arguments:        cur_state (enum status) - the current system state.
   Return Value:     (enum status) - the new system state (same as current
   		     state).

   Input:            None.
   Output:           The menu display is updated.

   Error Handling:   None.

   Algorithms:       None.
   Data Structures:  None.

   Global Variables: None.

   Author:           Glen George
   Last Modified:    Mar. 8, 1994

*/

enum status  menu_down(enum status cur_state)
{
    /* variables */
      /* none */



    /* go to the next menu entry */
    next_entry();


    /* return the current state */
    return  cur_state;

}




/*
   menu_left

   Description:      This function handles the <Left> key when in a menu.  It
                     invokes the left function for the current menu entry and
		     leaves the system state unchanged.

   Arguments:        cur_state (enum status) - the current system state.
   Return Value:     (enum status) - the new system state (same as current
   		     state).

   Input:            None.
   Output:           The menu display may be updated.

   Error Handling:   None.

   Algorithms:       None.
   Data Structures:  None.

   Global Variables: None.

   Author:           Glen George
   Last Modified:    Mar. 8, 1994

*/

enum status  menu_left(enum status cur_state)
{
    /* variables */
      /* none */



    /* invoke the <Left> key function for the current menu entry */
    menu_entry_left();


    /* return the current state */
    return  cur_state;

}




/*
   menu_right

   Description:      This function handles the <Right> key when in a menu.  It
                     invokes the right function for the current menu entry and
		     leaves the system state unchanged.

   Arguments:        cur_state (enum status) - the current system state.
   Return Value:     (enum status) - the new system state (same as current
   		     state).

   Input:            None.
   Output:           The menu display may be updated.

   Error Handling:   None.

   Algorithms:       None.
   Data Structures:  None.

   Global Variables: None.

   Author:           Glen George
   Last Modified:    Mar. 8, 1994

*/

enum status  menu_right(enum status cur_state)
{
    /* variables */
      /* none */



    /* invoke the <Right> key function for the current menu entry */
    menu_entry_right();


    /* return the current state */
    return  cur_state;

}
\end{lstlisting}

\subsection{keyint}
\lstset{language=[niosii]Assembler}
\begin{lstlisting}
################################################################################
#                                                                              #
#                             Key Interrupt Handler                            #
#                 	 Rotary Encoder and Menu Button Routines                   #
#                                   EE/CS 52                                   #
#                                                                              #
################################################################################


/*
 *  Albert Gural
 *  EE/CS 52
 *  TA: Dan Pipe-Mazo
 *
 *  File Description:	TODO
 *
 *  Table of Contents:	TODO
 *
 *  Revision History:
 *      02/09/2012  Dan Pipe-Mazo	Initial Revision.
 *		05/14/2014	Albert Gural	Begain writing assembly functions to handle
 *									keypress interrupts.
 *
 */

 /*  Local Include Files   */
#include "macros.m"
#include "pio.m"
#include "interfac.h"
#include "../osc_bsp/system.h"


.section  .text         #start code section


/*
 *  key_int_installer
 *
 *  Description: Installs a callback function for the key interrupts.
 *
 *  Arguments: (none)
 *
 *  Return Value: (none)
 *
 */

.global key_int_installer
.type	key_int_installer, @function

key_int_installer:
	SAVE

	# Enable all switch interrupts.
	movhi	r8, %hi(KEY_INPUT_BASE)
	ori		r8, r8, %lo(KEY_INPUT_BASE)
	movhi	r9, %hi(SWITCH_ALL)
	ori		r9, r9, %lo(SWITCH_ALL)
	stw		r9, PIO_IRQ_MASK(r8)

	# Install the interrupt handler
	mov		r4, r0
	movi	r5, KEY_INPUT_IRQ
	movhi	r6, %hi(key_handler)
	ori		r6, r6, %lo(key_handler)
	mov		r7, r0
	PUSH	r0
	call	alt_ic_isr_register
	POP		r0

key_int_installer_done:
	RESTORE
	ret


/*
 *  key_handler
 *
 *  Description: Callback function for what should happen when a key interrupt occurs.
 *  For menu keys, the key is saved until automatically requested by the main
 *  loop code. For the hardware-mimic keys, the desired key action is taken immediately.
 *
 *  Arguments: (none)
 *
 *  Return Value: (none)
 *
 */

.type key_handler,@function

key_handler:
	SAVE

	# Key should now be available. Update key_press.
	movi	r8, 1
	movia	r9, key_press
	stb		r8, (r9)

	# Clear interrupts.
	movhi	r8, %hi(KEY_INPUT_BASE)
	ori		r8, r8, %lo(KEY_INPUT_BASE)
	stw		r0, PIO_IRQ_MASK(r8)

	# Get the edge capture register.
	movhi	r8, %hi(KEY_INPUT_BASE)
	ori		r8, r8, %lo(KEY_INPUT_BASE)
	ldw		r8, PIO_EDGE_CAP(r8)

	# Check each bit (starting at 0) and see if set.
	movi	r9, 1
	movi	r11, 0

loop_keys:
	and		r10, r8, r9
	bne		r10, r0, key_lookup
	slli	r9, r9, 1
	addi	r11, r11, 1
	br		loop_keys

	# Once the key is found (r11), use the lookup table to set key_value.
key_lookup:
	movia	r8, key_map
	add		r8, r8, r11
	ldb		r8,	(r8)

	movia	r10, key_value
	stb		r8, (r10)

	# Do a lookup.
	movia	r4, set_trigger_normal
	movi	r12, 5
	beq		r11, r12, call_action
	movia	r4, set_trigger_single
	movi	r12, 6
	beq		r11, r12, call_action

	movia	r4, trg_slope_toggle
	movi	r12, 7
	beq		r11, r12, call_action
	movia	r4, trg_slope_toggle
	movi	r12, 8
	beq		r11, r12, call_action

	movia	r4, sweep_down
	movi	r12, 10
	beq		r11, r12, call_action
	movia	r4, sweep_up
	movi	r12, 11
	beq		r11, r12, call_action

	movia	r4, trg_level_up
	movi	r12, 16
	beq		r11, r12, call_action
	movia	r4, trg_level_down
	movi	r12, 17
	beq		r11, r12, call_action

	movia	r4, trg_delay_up
	movi	r12, 18
	beq		r11, r12, call_action
	movia	r4, trg_delay_down
	movi	r12, 19
	beq		r11, r12, call_action

key_lookup_cont:
	# Clear the edge capture register (write 1 to clear).
	movhi	r8, %hi(KEY_INPUT_BASE)
	ori		r8, r8, %lo(KEY_INPUT_BASE)
	movhi	r9, %hi(SWITCH_ALL)
	ori		r9, r9, %lo(SWITCH_ALL)
	stw		r9, PIO_EDGE_CAP(r8)

	# Re-enable interrupts.
	movhi	r8, %hi(KEY_INPUT_BASE)
	ori		r8, r8, %lo(KEY_INPUT_BASE)
	movhi	r9, %hi(SWITCH_ALL)
	ori		r9, r9, %lo(SWITCH_ALL)
	stw		r9, PIO_IRQ_MASK(r8)

key_handler_done:
	RESTORE
	ret

/* For certain keys, no menu action is desired. In these cases, the action
 * should be directly performed; not stored for later. This function calls a
 * particular action, saving and restoring the necessary registers. */
call_action:
	PUSH	r8
	PUSH	r9
	PUSH	r10
	PUSH	r11
	PUSH	r12
	PUSH	r13
	PUSH	r14
	PUSH	r15

	callr	r4

	POP		r15
	POP		r14
	POP		r13
	POP		r12
	POP		r11
	POP		r10
	POP		r9
	POP		r8
	br		key_lookup_cont


/*
 *  key_available
 *
 *  Description: Returns whether a key is available.
 *
 *  Arguments: (none)
 *
 *  Return Value:
 *  0 - no key available
 *  1 - key available
 *
 */

.global key_available
.type	key_available, @function

key_available:
	SAVE

	# Simply return the value in key_press.
	movia	r2, key_press
	ldb		r2, (r2)

key_available_done:
	RESTORE
	ret


/*
 *  get_key
 *
 *  Description: Waits until a key is available, then returns that key code.
 *
 *  Arguments: (none)
 *
 *  Return Value: key code value
 *
 */

.global	getkey
.type	getkey, @function

getkey:
	SAVE

	# Block until legal key arrives (which is also when key_press = TRUE).
	movia	r8, key_value
	ldb		r8, (r8)
	movi	r9, KEY_ILLEGAL
	beq		r8, r9, getkey

	# Get return value.
	movia	r2, key_value
	ldb		r2, (r2)

	# Update key_value with KEY_ILLEGAL.
	movia	r10, key_value
	stb		r9, (r10)

	# Update key_press with FALSE.
	movia	r10, key_press
	stb		r0, (r10)

getkey_done:
	RESTORE
	ret


/*
 *  key_map
 *
 *  Description: Lookup table for the key I/O bit to a key code value.
 *
 */

key_map:
			.byte	KEY_MENU
			.byte	KEY_UP
			.byte	KEY_DOWN
			.byte	KEY_LEFT
			.byte	KEY_RIGHT
			.byte	KEY_MENU
			.byte	KEY_MENU
			.byte	KEY_MENU
			.byte	KEY_MENU
			.byte	KEY_MENU
			.byte	KEY_MENU
			.byte	KEY_MENU
			.byte	KEY_MENU
			.byte	KEY_MENU
			.byte	KEY_MENU
			.byte	KEY_MENU
			.byte	KEY_MENU
			.byte	KEY_MENU
			.byte	KEY_MENU
			.byte	KEY_MENU
			.byte	KEY_ILLEGAL


.section  .data     #start data section

key_press:	.byte	0		# Gives whether a key has been pressed.
key_value:	.byte	0		# Gives the value of the pressed key.
\end{lstlisting}
