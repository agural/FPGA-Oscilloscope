\subsection{menu}
\begin{lstlisting}[language=C]
/****************************************************************************/
/*                                                                          */
/*                                  MENU.H                                  */
/*                              Menu Functions                              */
/*                               Include File                               */
/*                       Digital Oscilloscope Project                       */
/*                                 EE/CS 52                                 */
/*                                                                          */
/****************************************************************************/

/*
   This file contains the constants and function prototypes for the functions
   which deal with menus (defined in menu.c) for the Digital Oscilloscope
   project.


   Revision History:
      3/8/94   Glen George       Initial revision.
      3/13/94  Glen George       Updated comments.
      3/13/94  Glen George       Added definitions for SELECTED,
      				 OPTION_NORMAL, and OPTION_SELECTED.
*/



#ifndef  __MENU_H__
    #define  __MENU_H__


/* library include files */
  /* none */

/* local include files */
#include  "interfac.h"
#include  "scopedef.h"
#include  "lcdout.h"




/* constants */

/* menu size */
#define  MENU_WIDTH   16		/* menu width (in characters) */
#define  MENU_HEIGHT   7		/* menu height (in characters) */
#define  MENU_SIZE_X  (MENU_WIDTH * HORIZ_SIZE)  /* menu width (in pixels) */
#define  MENU_SIZE_Y  (MENU_HEIGHT * VERT_SIZE)  /* menu height (in pixels) */

/* menu position */
#define  MENU_X    (LCD_WIDTH - MENU_WIDTH - 1)  /* x position (in characters) */
#define  MENU_Y    0			         /* y position (in characters) */
#define  MENU_UL_X (MENU_X * HORIZ_SIZE)         /* x position (in pixels) */
#define  MENU_UL_Y (MENU_Y * VERT_SIZE)          /* y position (in pixels) */

/* menu colors */
#define  SELECTED         REVERSE	/* color for a selected menu entry */
#define  OPTION_SELECTED  NORMAL        /* color for a selected menu entry option */
#define  OPTION_NORMAL    NORMAL        /* color for an unselected menu entry option */

/* number of menu entries */
#define  NO_MENU_ENTRIES  (sizeof(menu) / sizeof(struct menu_item))




/* structures, unions, and typedefs */

/* data for an item in a menu */
struct menu_item  {  const char  *s;		/* string for menu entry */
		     int          h_off;	/* horizontal offset of entry */
		     int          opt_off;	/* horizontal offset of option setting */
		     void       (*display)(int, int, int);	/* option display function */
		  };




/* function declarations */

/* menu initialization function */
void  init_menu(void);

/* menu display functions */
void  clear_menu(void);		   /* clear the menu display */
void  display_menu(void);	   /* display the menu */
void  refresh_menu(void);	   /* refresh the menu */

/* menu update functions */
void  reset_menu(void);		   /* reset the menu to first entry */
void  next_entry(void);		   /* go to the next menu entry */
void  previous_entry(void);	   /* go to the previous menu entry */

/* menu entry functions */
void  menu_entry_left(void);	   /* do the <Left> key for the menu entry */
void  menu_entry_right(void);	   /* do the <Right> key for the menu entry */


#endif
\end{lstlisting}

\begin{lstlisting}[language=C]
/****************************************************************************/
/*                                                                          */
/*                                   MENU                                   */
/*                              Menu Functions                              */
/*                       Digital Oscilloscope Project                       */
/*                                 EE/CS 52                                 */
/*                                                                          */
/****************************************************************************/

/*
   This file contains the functions for processing menu entries for the
   Digital Oscilloscope project.  These functions take care of maintaining the
   menus and handling menu updates for the system.  The functions included
   are:
      clear_menu       - remove the menu from the display
      display_menu     - display the menu
      init_menu        - initialize menus
      menu_entry_left  - take care of <Left> key for a menu entry
      menu_entry_right - take care of <Right> key for a menu entry
      next_entry       - next menu entry
      previous_entry   - previous menu entry
      refresh_menu     - re-display the menu if currently being displayed
      reset_menu       - reset the current selection to the top of the menu

   The local functions included are:
      display_entry    - display a menu entry (including option setting)

   The locally global variable definitions included are:
      menu             - the menu
      menu_display     - whether or not the menu is currently displayed
      menu_entry       - the currently selected menu entry


   Revision History
      3/8/94   Glen George       Initial revision.
      3/9/94   Glen George       Changed position of const keyword in array
				 declarations involving pointers.
      3/13/94  Glen George       Updated comments.
      3/13/94  Glen George       Added display_entry function to output a menu
				 entry and option setting to the LCD (affects
				 many functions).
      3/13/94  Glen George       Changed calls to set_status due to changing
      				 enum scale_status definition.
      3/13/94  Glen George       No longer clear the menu area before
				 restoring the trace in clear_menu() (not
				 needed).
      3/17/97  Glen George       Updated comments.
      3/17/97  Glen George       Fixed minor bug in reset_menu().
      3/17/97  Glen George       When initializing the menu in init_menu(),
				 set the delay to MIN_DELAY instead of 0 and
				 trigger to a middle value instead of
				 MIN_TRG_LEVEL_SET.
      5/3/06   Glen George       Changed to a more appropriate constant in
                                 display_entry().
      5/3/06   Glen George       Updated comments.
      5/9/06   Glen George       Changed menus to handle a list for mode and
	                         scale (move up and down list), instead of
			         toggling values.
*/



/* library include files */
  /* none */

/* local include files */
#include  "scopedef.h"
#include  "lcdout.h"
#include  "menu.h"
#include  "menuact.h"
#include  "tracutil.h"




/* local function declarations */
void  display_entry(int, int);      /* display a menu entry and its setting */




/* locally global variables */
int  menu_display;           /* TRUE if menu is currently displayed */

const struct menu_item  menu[] =	    /* the menu */
    { { "Mode",    0, 4, display_mode      },
      { "Scale",   0, 5, display_scale     },
      { "Sweep",   0, 5, display_sweep     },
      { "Trigger", 0, 7, no_display        },
      { "Level",   2, 7, display_trg_level },
      { "Slope",   2, 7, display_trg_slope },
      { "Delay",   2, 7, display_trg_delay },
    };

int  menu_entry;		    /* currently selected menu entry */




/*
   init_menu

   Description:      This function initializes the menu routines.  It sets
                     the current menu entry to the first entry, indicates the
		     display is off, and initializes the options (and
		     hardware) to normal trigger mode, scale displayed, the
		     fastest sweep rate, a middle trigger level, positive
		     trigger slope, and minimum delay.  Finally, it displays
		     the menu.

   Arguments:        None.
   Return Value:     None.

   Input:            None.
   Output:           The menu is displayed.

   Error Handling:   None.

   Algorithms:       None.
   Data Structures:  None.

   Global Variables: menu_display - reset to FALSE.
   		     menu_entry   - reset to first entry (0).

   Author:           Glen George
   Last Modified:    Mar. 17, 1997

*/

void  init_menu(void)
{
    /* variables */
      /* none */



    /* set the menu parameters */
    menu_entry = 0;		/* first menu entry */
    menu_display = FALSE;	/* menu is not currently displayed (but it will be shortly) */


    /* set the scope (option) parameters */
    set_trigger_mode(NORMAL_TRIGGER);	/* normal triggering */
    set_scale(SCALE_AXES);		/* scale is axes */
    set_sweep(0);			/* first sweep rate */
    set_trg_level((MIN_TRG_LEVEL_SET + MAX_TRG_LEVEL_SET) / 2);	/* middle trigger level */
    set_trg_slope(SLOPE_POSITIVE);	/* positive slope */
    set_trg_delay(0);		/* default delay */


    /* now display the menu */
    display_menu();


    /* done initializing, return */
    return;

}




/*
   clear_menu

   Description:      This function removes the menu from the display.  The
                     trace under the menu is restored.  The flag menu_display,
		     is cleared, indicating the menu is no longer being
		     displayed.  Note: if the menu is not currently being
		     displayed this function does nothing.

   Arguments:        None.
   Return Value:     None.

   Input:            None.
   Output:           The menu if displayed, is removed and the trace under it
   		     is rewritten.

   Error Handling:   None.

   Algorithms:       None.
   Data Structures:  None.

   Global Variables: menu_display - checked and set to FALSE.

   Author:           Glen George
   Last Modified:    Mar. 13, 1994

*/

void  clear_menu(void)
{
    /* variables */
      /* none */



    /* check if the menu is currently being displayed */
//    if (menu_display)  {
//
//        /* menu is being displayed - turn it off and restore the trace in that area */
//	restore_menu_trace();
//    }


    /* no longer displaying the menu */
    menu_display = FALSE;


    /* all done, return */
    return;

}




/*
   display_menu

   Description:      This function displays the menu.  The trace under the
                     menu is overwritten (but it was saved).  The flag
		     menu_display, is also set, indicating the menu is
		     currently being displayed.  Note: if the menu is already
		     being displayed this function does not redisplay it.

   Arguments:        None.
   Return Value:     None.

   Input:            None.
   Output:           The menu is displayed.

   Error Handling:   None.

   Algorithms:       None.
   Data Structures:  None.

   Global Variables: menu_display - set to TRUE.
   		     menu_entry   - used to highlight currently selected entry.

   Author:           Glen George
   Last Modified:    Mar. 13, 1994

*/

void  display_menu(void)
{
    /* variables */
    int  i;		/* loop index */



    /* check if the menu is currently being displayed */
    if (!menu_display)  {

        /* menu is not being displayed - turn it on */
	/* display it entry by entry */
	for (i = 0; i < NO_MENU_ENTRIES; i++)  {

	    /* display this entry - check if it should be highlighted */
	    if (i == menu_entry)
	        /* currently selected entry - highlight it */
	        display_entry(i, TRUE);
	    else
	        /* not the currently selected entry - "normal video" */
	        display_entry(i, FALSE);
        }
    }


    /* now are displaying the menu */
    menu_display = TRUE;


    /* all done, return */
    return;

}




/*
   refresh_menu

   Description:      This function displays the menu if it is currently being
		     displayed.  The trace under the menu is overwritten (but
		     it was already saved).

   Arguments:        None.
   Return Value:     None.

   Input:            None.
   Output:           The menu is displayed.

   Error Handling:   None.

   Algorithms:       None.
   Data Structures:  None.

   Global Variables: menu_display - determines if menu should be displayed.

   Author:           Glen George
   Last Modified:    Mar. 8, 1994

*/

void  refresh_menu(void)
{
    /* variables */
      /* none */



    /* check if the menu is currently being displayed */
    if (menu_display)  {

    	/* menu is currently being displayed - need to refresh it */
	/* do this by turning off the display, then forcing it back on */
	menu_display = FALSE;
	display_menu();
    }


    /* refreshed the menu if it was displayed, now return */
    return;

}




/*
   reset_menu

   Description:      This function resets the current menu selection to the
                     first menu entry.  If the menu is currently being
		     displayed the display is updated.

   Arguments:        None.
   Return Value:     None.

   Input:            None.
   Output:           The menu display is updated if it is being displayed.

   Error Handling:   None.

   Algorithms:       None.
   Data Structures:  None.

   Global Variables: menu_display - checked to see if menu is displayed.
   		     menu_entry   - reset to 0 (first entry).

   Author:           Glen George
   Last Modified:    Mar. 17, 1997

*/

void  reset_menu(void)
{
    /* variables */
      /* none */



    /* check if the menu is currently being displayed */
    if (menu_display)  {

        /* menu is being displayed */
	/* remove highlight from currently selected entry */
	display_entry(menu_entry, FALSE);
    }


    /* reset the currently selected entry */
    menu_entry = 0;


    /* finally, highlight the first entry if the menu is being displayed */
    if (menu_display)
	display_entry(menu_entry, TRUE);



    /* all done, return */
    return;

}




/*
   next_entry

   Description:      This function changes the current menu selection to the
                     next menu entry.  If the current selection is the last
		     entry in the menu, it is not changed.  If the menu is
		     currently being displayed, the display is updated.

   Arguments:        None.
   Return Value:     None.

   Input:            None.
   Output:           The menu display is updated if it is being displayed and
   		     the entry selected changes.

   Error Handling:   None.

   Algorithms:       None.
   Data Structures:  None.

   Global Variables: menu_display - checked to see if menu is displayed.
   		     menu_entry   - updated to a new entry (if not at end).

   Author:           Glen George
   Last Modified:    Mar. 13, 1994

*/

void  next_entry(void)
{
    /* variables */
      /* none */



    /* only update if not at end of the menu */
    if (menu_entry < (NO_MENU_ENTRIES - 1))  {

        /* not at the end of the menu */

	/* turn off current entry if displaying */
	if (menu_display)
            /* displaying menu - turn off currently selected entry */
	    display_entry(menu_entry, FALSE);

	/* update the menu entry to the next one */
	menu_entry++;

	/* now highlight this entry if displaying the menu */
	if (menu_display)
            /* displaying menu - highlight newly selected entry */
	    display_entry(menu_entry, TRUE);
    }


    /* all done, return */
    return;

}




/*
   previous_entry

   Description:      This function changes the current menu selection to the
                     previous menu entry.  If the current selection is the 
		     first entry in the menu, it is not changed.  If the menu
		     is currently being displayed, the display is updated.

   Arguments:        None.
   Return Value:     None.

   Input:            None.
   Output:           The menu display is updated if it is being displayed and
   		     the currently selected entry changes.

   Error Handling:   None.

   Algorithms:       None.
   Data Structures:  None.

   Global Variables: menu_display - checked to see if menu is displayed.
   		     menu_entry   - updated to a new entry (if not at start).

   Author:           Glen George
   Last Modified:    Mar. 13, 1994

*/

void  previous_entry(void)
{
    /* variables */
      /* none */



    /* only update if not at the start of the menu */
    if (menu_entry > 0)  {

        /* not at the start of the menu */

	/* turn off current entry if displaying */
	if (menu_display)
            /* displaying menu - turn off currently selected entry */
	    display_entry(menu_entry, FALSE);

	/* update the menu entry to the previous one */
	menu_entry--;

	/* now highlight this entry if displaying the menu */
	if (menu_display)
            /* displaying menu - highlight newly selected entry */
	    display_entry(menu_entry, TRUE);

    }


    /* all done, return */
    return;

}




/*
   menu_entry_left

   Description:      This function handles the <Left> key for the current menu
                     selection.  It does this by doing a table lookup on the
		     current menu selection.

   Arguments:        None.
   Return Value:     None.

   Input:            None.
   Output:           The menu display is updated if it is being displayed and
   		     the <Left> key causes a change to the display.

   Error Handling:   None.

   Algorithms:       Table lookup is used to determine what to do for the
   		     input key.
   Data Structures:  An array holds the table of key processing routines.

   Global Variables: menu_entry - used to select the processing function.

   Author:           Glen George
   Last Modified:    May 9, 2006

*/

void  menu_entry_left(void)
{
    /* variables */

    /* key processing functions */
    void  (* const process[])(void) =
       /*  Mode            Scale             Sweep           Trigger      */
        {  mode_down,      scale_down,       sweep_down,     trace_rearm,
           trg_level_down, trg_slope_toggle, trg_delay_down               };
       /*  Level           Slope             Delay                        */



    /* invoke the appropriate <Left> key function */
    process[menu_entry]();

    /* if displaying menu entries, display the new value */
    /* note: since it is being changed - know this option is selected */
    if (menu_display)  {
        menu[menu_entry].display((MENU_X + menu[menu_entry].opt_off),
    			         (MENU_Y + menu_entry), OPTION_SELECTED);
    }


    /* all done, return */
    return;

}




/*
   menu_entry_right

   Description:      This function handles the <Right> key for the current
                     menu selection.  It does this by doing a table lookup on
		     the current menu selection.

   Arguments:        None.
   Return Value:     None.

   Input:            None.
   Output:           The menu display is updated if it is being displayed and
   		     the <Right> key causes a change to the display.

   Error Handling:   None.

   Algorithms:       Table lookup is used to determine what to do for the
   		     input key.
   Data Structures:  An array holds the table of key processing routines.

   Global Variables: menu       - used to display the new menu value.
   		     menu_entry - used to select the processing function.

   Author:           Glen George
   Last Modified:    May 9, 2006

*/

void  menu_entry_right(void)
{
    /* variables */

    /* key processing functions */
    void  (* const process[])(void) =
       /*  Mode          Scale             Sweep           Trigger      */
        {  mode_up    ,  scale_up,     sweep_up,       trace_rearm,
           trg_level_up, trg_slope_toggle, trg_delay_up                 };
       /*  Level         Slope             Delay                        */



    /* invoke the appropriate <Right> key function */
    process[menu_entry]();

    /* if displaying menu entries, display the new value */
    /* note: since it is being changed - know this option is selected */
    if (menu_display)  {
        menu[menu_entry].display((MENU_X + menu[menu_entry].opt_off),
    			         (MENU_Y + menu_entry), OPTION_SELECTED);
    }


    /* all done, return */
    return;

}




/*
   display_entry

   Description:      This function displays the passed menu entry and its
   		     current option setting.  If the second argument is TRUE
		     it displays them with color SELECTED and OPTION_SELECTED
		     respectively.  If the second argument is FALSE it
		     displays the menu entry with color NORMAL and the option
		     setting with color OPTION_NORMAL.

   Arguments:        entry (int)    - menu entry to be displayed.
   		     selected (int) - whether or not the menu entry is
		     		      currently selected (determines the color
				      with which the entry is output).
   Return Value:     None.

   Input:            None.
   Output:           The menu entry is output to the LCD.

   Error Handling:   None.

   Algorithms:       None.
   Data Structures:  None.

   Global Variables: menu - used to display the menu entry.

   Author:           Glen George
   Last Modified:    Aug. 13, 2004

*/

void  display_entry(int entry, int selected)
{
    /* variables */
      /* none */



    /* output the menu entry with the appropriate color */
    plot_string((MENU_X + menu[entry].h_off), (MENU_Y + entry), menu[entry].s,
    		(selected ? SELECTED : NORMAL));
    /* also output the menu option with the appropriate color */
    menu[entry].display((MENU_X + menu[entry].opt_off), (MENU_Y + entry),
    			(selected ? OPTION_SELECTED : OPTION_NORMAL));


    /* all done outputting this menu entry - return */
    return;

}
\end{lstlisting}

\subsection{menuact}
\begin{lstlisting}[language=C]
/****************************************************************************/
/*                                                                          */
/*                                MENUACT.H                                 */
/*                          Menu Action Functions                           */
/*                               Include File                               */
/*                       Digital Oscilloscope Project                       */
/*                                 EE/CS 52                                 */
/*                                                                          */
/****************************************************************************/

/*
   This file contains the constants and function prototypes for the functions
   which carry out menu actions and display and initialize menu settings for
   the Digital Oscilloscope project (the functions are defined in menuact.c).


   Revision History:
      3/8/94   Glen George       Initial revision.
      3/13/94  Glen George       Updated comments.
      3/13/94  Glen George       Changed definition of enum scale_type (was
      				 enum scale_status).
      3/10/95  Glen George       Changed MAX_TRG_LEVEL_SET (maximum trigger
      				 level) to 127 to match specification.
      3/17/97  Glen George       Updated comments.
      5/3/06   Glen George       Updated comments.
      5/9/06   Glen George       Added a new mode (AUTO_TRIGGER) and a new
                                 scale (SCALE_GRID).
      5/9/06   Glen George       Added menu functions for mode and scale to
                                 move up and down a list instead of just
				 toggling the selection.
      5/9/06   Glen George       Added declaration for the accessor to the
                                 current trigger mode (get_trigger_mode).
*/



#ifndef  __MENUACT_H__
    #define  __MENUACT_H__


/* library include files */
  /* none */

/* local include files */
#include  "interfac.h"
#include  "lcdout.h"




/* constants */

/* min and max trigger level settings */
#define  MIN_TRG_LEVEL_SET   0
#define  MAX_TRG_LEVEL_SET   127

/* number of different sweep rates */
#define  NO_SWEEP_RATES     (sizeof(sweep_rates) / sizeof(struct sweep_info))




/* structures, unions, and typedefs */

/* types of triggering modes */
enum trigger_type  {  NORMAL_TRIGGER,		/* normal triggering */
		      AUTO_TRIGGER,		/* automatic triggering */
		      ONESHOT_TRIGGER		/* one-shot triggering */
		   };

/* types of displayed scales */
enum scale_type    {  SCALE_NONE,		/* no scale is displayed */
		      SCALE_AXES,		/* scale is a set of axes */
		      SCALE_GRID		/* scale is a grid */
		   };

/* types of trigger slopes */
enum slope_type    {  SLOPE_POSITIVE,		/* positive trigger slope */
		      SLOPE_NEGATIVE		/* negative trigger slope */
		   };

/* sweep rate information */
struct sweep_info  {  long int     sample_rate;    /* sample rate */
		      const char  *s;		   /* sweep rate string */
		   };




/* function declarations */

/* menu option actions */
void  no_menu_action(void);    /* no action to perform */
void  mode_down(void);         /* change to the "next" trigger mode */
void  mode_up(void);           /* change to the "previous" trigger mode */
void  scale_down(void);        /* change to the "next" scale type */
void  scale_up(void);          /* change to the "previous" scale type */
void  sweep_down(void);        /* decrease the sweep rate */
void  sweep_up(void);          /* increase the sweep rate */
void  trg_level_down(void);    /* decrease the trigger level */
void  trg_level_up(void);      /* increase the trigger level */
void  trg_slope_toggle(void);  /* toggle the trigger slope */
void  trg_delay_down(void);    /* decrease the trigger delay */
void  trg_delay_up(void);      /* increase the trigger delay */

/* option accessor routines */
enum trigger_type  get_trigger_mode(void);  /* get the current trigger mode */

/* option initialization routines */
void  set_trigger_mode(enum trigger_type);  /* set the trigger mode */
void  set_trigger_normal(void);  			/* normal trigger mode */
void  set_trigger_auto(void);  				/* auto trigger mode */
void  set_trigger_single(void);  			/* single trigger mode */
void  set_scale(enum scale_type);           /* set the scale type */
void  set_sweep(int);         		    /* set the sweep rate */
void  set_trg_level(int);     		    /* set the trigger level */
void  set_trg_slope(enum slope_type);       /* set the trigger slope */
void  set_trg_delay(long int);     	    /* set the tigger delay */

/* option display routines */
void  no_display(int, int, int);	 /* no option setting to display */
void  display_mode(int, int, int);       /* display trigger mode */
void  display_scale(int, int, int);      /* display the scale type */
void  display_sweep(int, int, int);      /* display the sweep rate */
void  display_trg_level(int, int, int);  /* display the trigger level */
void  display_trg_slope(int, int, int);  /* display the trigger slope */
void  display_trg_delay(int, int, int);  /* display the tigger delay */


#endif
\end{lstlisting}

\begin{lstlisting}[language=C]
/****************************************************************************/
/*                                                                          */
/*                                 MENUACT                                  */
/*                          Menu Action Functions                           */
/*                       Digital Oscilloscope Project                       */
/*                                 EE/CS 52                                 */
/*                                                                          */
/****************************************************************************/

/*
   This file contains the functions for carrying out menu actions for the
   Digital Oscilloscope project.  These functions are invoked when the <Left>
   or <Right> key is pressed for a menu item.  Also included are the functions
   for displaying the current menu option selection.  The functions included
   are:
      display_mode      - display trigger mode
      display_scale     - display the scale type
      display_sweep     - display the sweep rate
      display_trg_delay - display the tigger delay
      display_trg_level - display the trigger level
      display_trg_slope - display the trigger slope
      get_trigger_mode  - get the current trigger mode
      mode_down         - go to the "next" trigger mode
      mode_up           - go to the "previous" trigger mode
      no_display        - nothing to display for option setting
      no_menu_action    - no action to perform for <Left> or <Right> key
      scale_down        - go to the "next" scale type
      scale_up          - go to the "previous" scale type
      set_scale         - set the scale type
      set_sweep         - set the sweep rate
      set_trg_delay     - set the tigger delay
      set_trg_level     - set the trigger level
      set_trg_slope     - set the trigger slope
      set_trigger_mode  - set the trigger mode
      sweep_down        - decrease the sweep rate
      sweep_up          - increase the sweep rate
      trg_delay_down    - decrease the trigger delay
      trg_delay_up      - increase the trigger delay
      trg_level_down    - decrease the trigger level
      trg_level_up      - increase the trigger level
      trg_slope_toggle  - toggle the trigger slope between "+" and "-"

   The local functions included are:
      adjust_trg_delay  - adjust the trigger delay for a new sweep rate
      cvt_num_field     - converts a numeric field value to a string

   The locally global variable definitions included are:
      delay         - current trigger delay
      level         - current trigger level
      scale         - current display scale type
      slope         - current trigger slope
      sweep         - current sweep rate
      sweep_rates   - table of information on possible sweep rates
      trigger_mode  - current triggering mode


   Revision History
      3/8/94   Glen George       Initial revision.
      3/13/94  Glen George       Updated comments.
      3/13/94  Glen George       Changed all arrays of constant strings to be
      				 so compiler generates correct code.
      3/13/94  Glen George       Changed scale to type enum scale_type and
      				 output the selection as "None" or "Axes".
				 This will allow for easier future expansion.
      3/13/94  Glen George       Changed name of set_axes function (in
      				 tracutil.c) to set_display_scale.
      3/10/95  Glen George       Changed calculation of displayed trigger
      				 level to use constants MIN_TRG_LEVEL_SET and
				 MAX_TRG_LEVEL_SET to get the trigger level
				 range.
      3/17/97  Glen George       Updated comments.
      5/3/06   Glen George       Changed sweep definitions to include new
      				 sweep rates of 100 ns, 200 ns, 500 ns, and
			         1 us and updated functions to handle these
				 new rates.
      5/9/06   Glen George       Added new a triggering mode (automatic
                                 triggering) and a new scale (grid) and
                                 updated functions to implement these options.
      5/9/06   Glen George       Added functions for setting the triggering
                                 mode and scale by going up and down the list
                                 of possibilities instead of just toggling
                                 between one of two possibilities (since there
				 are more than two now).
      5/9/06   Glen George       Added accessor function (get_trigger_mode)
                                 to be able to get the current trigger mode.
*/



/* library include files */
  /* none */

/* local include files */
#include  "interfac.h"
#include  "scopedef.h"
#include  "lcdout.h"
#include  "menuact.h"
#include  "tracutil.h"




/* local function declarations */
void  adjust_trg_delay(int, int);       /* adjust the trigger delay for new sweep */
void  cvt_num_field(long int, char *);	/* convert a number to a string */




/* locally global variables */

/* trace parameters */
enum trigger_type   trigger_mode;	/* current triggering mode */
enum scale_type     scale;	 		/* current scale type */
int		   			sweep;        	/* sweep rate index */
int		   			level;	 		/* current trigger level */
enum slope_type     slope;	 		/* current trigger slope */
long int	        delay;	 		/* current trigger delay */

/* sweep rate information */
const struct sweep_info  sweep_rates[] =
    { {200000000L, " 5 ns  " },
      {100000000L, " 10 ns " },
      { 50000000L, " 20 ns " },
      { 20000000L, " 50 ns " },
      { 10000000L, " 100 ns" },
      {  5000000L, " 200 ns" },
      {  2000000L, " 500 ns" },
      {  1000000L, " 1 \004s  " },
      {   500000L, " 2 \004s  " },
      {   200000L, " 5 \004s  " },
      {   100000L, " 10 \004s " },
      {    50000L, " 20 \004s " },
      {    20000L, " 50 \004s " },
      {    10000L, " 100 \004s" },
      {     5000L, " 200 \004s" },
      {     2000L, " 500 \004s" },
      {     1000L, " 1 ms  "    },
      {      500L, " 2 ms  "    },
      {      200L, " 5 ms  "    },
      {      100L, " 10 ms "    },
      {       50L, " 20 ms "    } };




/*
   no_menu_action

   Description:      This function handles a menu action when there is nothing
                     to be done.  It just returns.

   Arguments:        None.
   Return Value:     None.

   Input:            None.
   Output:           None.

   Error Handling:   None.

   Algorithms:       None.
   Data Structures:  None.

   Global Variables: None.

   Author:           Glen George
   Last Modified:    Mar. 8, 1994

*/

void  no_menu_action()
{
    /* variables */
      /* none */



    /* nothing to do - return */
    return;

}




/*
   no_display

   Description:      This function handles displaying a menu option's setting
                     when there is nothing to display.  It just returns,
		     ignoring all arguments.

   Arguments:        x_pos (int) - x position (in character cells) at which to
   				   display the menu option (not used).
   		     y_pos (int) - y position (in character cells) at which to
   				   display the menu option (not used).
		     style (int) - style with which to display the menu option
		     		   (not used).
   Return Value:     None.

   Input:            None.
   Output:           None.

   Error Handling:   None.

   Algorithms:       None.
   Data Structures:  None.

   Global Variables: None.

   Author:           Glen George
   Last Modified:    Mar. 8, 1994

*/

void  no_display(int x_pos, int y_pos, int style)
{
    /* variables */
      /* none */



    /* nothing to do - return */
    return;

}




/*
   set_trigger_mode

   Description:      This function sets the triggering mode to the passed
                     value.

   Arguments:        m (enum trigger_type) - mode to which to set the
   					     triggering mode.
   Return Value:     None.

   Input:            None.
   Output:           None.

   Error Handling:   None.

   Algorithms:       None.
   Data Structures:  None.

   Global Variables: trigger_mode - initialized to the passed value.

   Author:           Glen George
   Last Modified:    Mar. 8, 1994

*/

void  set_trigger_mode(enum trigger_type m)
{
    /* variables */
      /* none */



    /* set the trigger mode */
    trigger_mode = m;

    /* set the new mode */
    set_mode(trigger_mode);


    /* all done setting the trigger mode - return */
    return;

}




/*
   set_trigger_normal

   Description:      This function sets the triggering mode to normal.

   Arguments:        None.
   Return Value:     None.

   Input:            None.
   Output:           None.

   Error Handling:   None.

   Algorithms:       None.
   Data Structures:  None.

   Global Variables: trigger_mode - initialized to the passed value.

   Author:           Albert Gural
   Last Modified:    Jun. 13, 2014

*/

void  set_trigger_normal()
{
    /* variables */
      /* none */



    /* set the trigger mode */
    trigger_mode = NORMAL_TRIGGER;

    /* set the new mode */
    set_mode(trigger_mode);


    /* all done setting the trigger mode - return */
    return;

}




/*
   set_trigger_auto

   Description:      This function sets the triggering mode to auto.

   Arguments:        None.
   Return Value:     None.

   Input:            None.
   Output:           None.

   Error Handling:   None.

   Algorithms:       None.
   Data Structures:  None.

   Global Variables: trigger_mode - initialized to the passed value.

   Author:           Albert Gural
   Last Modified:    Jun. 13, 2014

*/

void  set_trigger_auto()
{
    /* variables */
      /* none */



    /* set the trigger mode */
    trigger_mode = AUTO_TRIGGER;

    /* set the new mode */
    set_mode(trigger_mode);


    /* all done setting the trigger mode - return */
    return;

}




/*
   set_trigger_single

   Description:      This function sets the triggering mode to single/one-shot.

   Arguments:        None.
   Return Value:     None.

   Input:            None.
   Output:           None.

   Error Handling:   None.

   Algorithms:       None.
   Data Structures:  None.

   Global Variables: trigger_mode - initialized to the passed value.

   Author:           Albert Gural
   Last Modified:    Jun. 13, 2014

*/

void  set_trigger_single()
{
    /* variables */
      /* none */



    /* set the trigger mode */
    trigger_mode = ONESHOT_TRIGGER;

    /* set the new mode */
    set_mode(trigger_mode);


    /* all done setting the trigger mode - return */
    return;

}




/*
   get_trigger_mode

   Description:      This function returns the current triggering mode.

   Arguments:        None.
   Return Value:     (enum trigger_type) - current triggering mode.

   Input:            None.
   Output:           None.

   Error Handling:   None.

   Algorithms:       None.
   Data Structures:  None.

   Global Variables: trigger_mode - value is returned (not changed).

   Author:           Glen George
   Last Modified:    May 9, 2006

*/

enum trigger_type  get_trigger_mode()
{
    /* variables */
      /* none */



    /* return the current trigger mode */
    return  trigger_mode;

}




/*
   mode_down

   Description:      This function handles moving down the list of trigger
                     modes.  It changes to the "next" triggering mode and
                     sets that as the current mode.

   Arguments:        None.
   Return Value:     None.

   Input:            None.
   Output:           None.

   Error Handling:   None.

   Algorithms:       None.
   Data Structures:  None.

   Global Variables: trigger_mode - changed to "next" trigger mode.

   Author:           Glen George
   Last Modified:    May 9, 2006

*/

void  mode_down()
{
    /* variables */
      /* none */



    /* move to the "next" triggering mode */
    if (trigger_mode == NORMAL_TRIGGER)
        trigger_mode = AUTO_TRIGGER;
    else if (trigger_mode == AUTO_TRIGGER)
        trigger_mode = ONESHOT_TRIGGER;
    else
        trigger_mode = NORMAL_TRIGGER;

    /* set the new mode */
    set_mode(trigger_mode);


    /* all done with the trigger mode - return */
    return;

}




/*
   mode_up

   Description:      This function handles moving up the list of trigger
                     modes.  It changes to the "previous" triggering mode and
                     sets that as the current mode.

   Arguments:        None.
   Return Value:     None.

   Input:            None.
   Output:           None.

   Error Handling:   None.

   Algorithms:       None.
   Data Structures:  None.

   Global Variables: trigger_mode - changed to "previous" trigger mode.

   Author:           Glen George
   Last Modified:    May 9, 2006

*/

void  mode_up()
{
    /* variables */
      /* none */



    /* move to the "previous" triggering mode */
    if (trigger_mode == NORMAL_TRIGGER)
        trigger_mode = ONESHOT_TRIGGER;
    else if (trigger_mode == AUTO_TRIGGER)
        trigger_mode = NORMAL_TRIGGER;
    else
        trigger_mode = AUTO_TRIGGER;

    /* set the new mode */
    set_mode(trigger_mode);


    /* all done with the trigger mode - return */
    return;

}




/*
   display_mode

   Description:      This function displays the current triggering mode at the
                     passed position, in the passed style.

   Arguments:        x_pos (int) - x position (in character cells) at which to
   				   display the trigger mode.
   		     y_pos (int) - y position (in character cells) at which to
   				   display the trigger mode.
		     style (int) - style with which to display the trigger
		     		   mode.
   Return Value:     None.

   Input:            None.
   Output:           The trigger mode is displayed at the passed position on
   		     the screen.

   Error Handling:   None.

   Algorithms:       None.
   Data Structures:  None.

   Global Variables: trigger_mode - determines which string is displayed.

   Author:           Glen George
   Last Modified:    May 9, 2006

*/

void  display_mode(int x_pos, int y_pos, int style)
{
    /* variables */

    /* the mode strings (must match enumerated type) */
    const char * const  modes[] =  {  " Normal   ",
                                             " Automatic",
                                             " One-Shot "  };



    /* display the trigger mode */
    plot_string(x_pos, y_pos, modes[trigger_mode], style);


    /* all done displaying the trigger mode - return */
    return;

}




/*
   set_scale

   Description:      This function sets the scale type to the passed value.

   Arguments:        s (enum scale_type) - scale type to which to initialize
   					   the scale status.
   Return Value:     None.

   Input:            None.
   Output:           The new trace display is updated with the new scale.

   Error Handling:   None.

   Algorithms:       None.
   Data Structures:  None.

   Global Variables: scale - initialized to the passed value.

   Author:           Glen George
   Last Modified:    Mar. 13, 1994

*/

void  set_scale(enum scale_type s)
{
    /* variables */
      /* none */



    /* set the scale type */
    scale = s;

    /* output the scale appropriately */
    set_display_scale(scale);


    /* all done setting the scale type - return */
    return;

}




/*
   scale_down

   Description:      This function handles moving down the list of scale
                     types.  It changes to the "next" type of scale and sets
   		     this as the current scale type.

   Arguments:        None.
   Return Value:     None.

   Input:            None.
   Output:           The new scale is output to the trace display.

   Error Handling:   None.

   Algorithms:       None.
   Data Structures:  None.

   Global Variables: scale - changed to the "next" scale type.

   Author:           Glen George
   Last Modified:    May 9, 2006

*/

void  scale_down()
{
    /* variables */
      /* none */



    /* change to the "next" scale type */
    if (scale == SCALE_NONE)
        scale = SCALE_AXES;
    else if (scale == SCALE_AXES)
        scale = SCALE_GRID;
    else
        scale = SCALE_NONE;

    /* set the scale type */
    set_display_scale(scale);


    /* all done with toggling the scale type - return */
    return;

}




/*
   scale_up

   Description:      This function handles moving up the list of scale types.
                     It changes to the "previous" type of scale and sets this
   		     as the current scale type.

   Arguments:        None.
   Return Value:     None.

   Input:            None.
   Output:           The new scale is output to the trace display.

   Error Handling:   None.

   Algorithms:       None.
   Data Structures:  None.

   Global Variables: scale - changed to the "previous" scale type.

   Author:           Glen George
   Last Modified:    May 9, 2006

*/

void  scale_up()
{
    /* variables */
      /* none */



    /* change to the "previous" scale type */
    if (scale == SCALE_NONE)
        scale = SCALE_GRID;
    else if (scale == SCALE_AXES)
        scale = SCALE_NONE;
    else
        scale = SCALE_AXES;

    /* set the scale type */
    set_display_scale(scale);


    /* all done with toggling the scale type - return */
    return;

}




/*
   display_scale

   Description:      This function displays the current scale type at the
                     passed position, in the passed style.

   Arguments:        x_pos (int) - x position (in character cells) at which to
   				   display the scale type.
   		     y_pos (int) - y position (in character cells) at which to
   				   display the scale type.
		     style (int) - style with which to display the scale type.
   Return Value:     None.

   Input:            None.
   Output:           The scale type is displayed at the passed position on the
   		     display.

   Error Handling:   None.

   Algorithms:       None.
   Data Structures:  None.

   Global Variables: scale - determines which string is displayed.

   Author:           Glen George
   Last Modified:    Mar. 13, 1994

*/

void  display_scale(int x_pos, int y_pos, int style)
{
    /* variables */

    /* the scale type strings (must match enumerated type) */
    const char * const  scale_stat[] =  {  " None",
                                                  " Axes",
                                                  " Grid"  };



    /* display the scale status */
    plot_string(x_pos, y_pos, scale_stat[scale], style);


    /* all done displaying the scale status - return */
    return;

}




/*
   set_sweep

   Description:      This function sets the sweep rate to the passed value.
                     The passed value gives the sweep rate to choose from the
		     list of sweep rates (it gives the list index).

   Arguments:        s (int) - index into the list of sweep rates to which to
   			       set the current sweep rate.
   Return Value:     None.

   Input:            None.
   Output:           None.

   Error Handling:   The passed index is not checked for validity.

   Algorithms:       None.
   Data Structures:  None.

   Global Variables: sweep - initialized to the passed value.

   Author:           Glen George
   Last Modified:    Mar. 8, 1994

*/

void  set_sweep(int s)
{
    /* variables */
    int  sample_size;		/* sample size for this sweep rate */



    /* set the new sweep rate */
    sweep = s;

    /* set the sweep rate for the hardware */
    sample_size = set_sample_rate(sweep_rates[sweep].sample_rate);
    /* also set the sample size for the trace capture */
    set_trace_size(sample_size);


    /* all done initializing the sweep rate - return */
    return;

}




/*
   sweep_down

   Description:      This function handles decreasing the current sweep rate.
   		     The new sweep rate (and sample size) is sent to the
		     hardware (and trace routines).  If an attempt is made to
		     lower the sweep rate below the minimum value it is not
		     changed.  This routine also updates the sweep delay based
		     on the new sweep rate (to keep the delay time constant).

   Arguments:        None.
   Return Value:     None.

   Input:            None.
   Output:           None.

   Error Handling:   None.

   Algorithms:       None.
   Data Structures:  None.

   Global Variables: sweep - decremented if not already 0.
   		     delay - increased to keep delay time constant.

   Known Bugs:       The updated delay time is not displayed.  Since the time
   		     is typically only rounded to the new sample rate, this is
		     not a major problem.

   Author:           Glen George
   Last Modified:    Mar. 8, 1994

*/

void  sweep_down()
{
    /* variables */
    int  sample_size;		/* sample size for the new sweep rate */



    /* decrease the sweep rate, if not already the minimum */
    if (sweep > 0)  {
        /* not at minimum, adjust delay for new sweep */
	adjust_trg_delay(sweep, (sweep - 1));
	/* now set new sweep rate */
        sweep--;
    }

    /* set the sweep rate for the hardware */
    sample_size = set_sample_rate(sweep_rates[sweep].sample_rate);
    /* also set the sample size for the trace capture */
    set_trace_size(sample_size);


    /* all done with lowering the sweep rate - return */
    return;

}




/*
   sweep_up

   Description:      This function handles increasing the current sweep rate.
   		     The new sweep rate (and sample size) is sent to the
		     hardware (and trace routines).  If an attempt is made to
		     raise the sweep rate above the maximum value it is not
		     changed.  This routine also updates the sweep delay based
		     on the new sweep rate (to keep the delay time constant).

   Arguments:        None.
   Return Value:     None.

   Input:            None.
   Output:           None.

   Error Handling:   None.

   Algorithms:       None.
   Data Structures:  None.

   Global Variables: sweep - incremented if not already the maximum value.
   		     delay - decreased to keep delay time constant.

   Known Bugs:       The updated delay time is not displayed.  Since the time
   		     is typically only rounded to the new sample rate, this is
		     not a major problem.

   Author:           Glen George
   Last Modified:    Mar. 8, 1994

*/

void  sweep_up()
{
    /* variables */
    int  sample_size;		/* sample size for the new sweep rate */



    /* increase the sweep rate, if not already the maximum */
    if (sweep < (NO_SWEEP_RATES - 1))  {
        /* not at maximum, adjust delay for new sweep */
	adjust_trg_delay(sweep, (sweep + 1));
	/* now set new sweep rate */
        sweep++;
    }

    /* set the sweep rate for the hardware */
    sample_size = set_sample_rate(sweep_rates[sweep].sample_rate);
    /* also set the sample size for the trace capture */
    set_trace_size(sample_size);


    /* all done with raising the sweep rate - return */
    return;

}




/*
   display_sweep

   Description:      This function displays the current sweep rate at the
                     passed position, in the passed style.

   Arguments:        x_pos (int) - x position (in character cells) at which to
   				   display the sweep rate.
   		     y_pos (int) - y position (in character cells) at which to
   				   display the sweep rate.
		     style (int) - style with which to display the sweep rate.
   Return Value:     None.

   Input:            None.
   Output:           The sweep rate is displayed at the passed position on the
   		     display.

   Error Handling:   None.

   Algorithms:       None.
   Data Structures:  None.

   Global Variables: sweep - determines which string is displayed.

   Author:           Glen George
   Last Modified:    Mar. 8, 1994

*/

void  display_sweep(int x_pos, int y_pos, int style)
{
    /* variables */
      /* none */



    /* display the sweep rate */
    plot_string(x_pos, y_pos, sweep_rates[sweep].s, style);


    /* all done displaying the sweep rate - return */
    return;

}




/*
   set_trg_level

   Description:      This function sets the trigger level to the passed value.

   Arguments:        l (int) - value to which to set the trigger level.
   Return Value:     None.

   Input:            None.
   Output:           None.

   Error Handling:   The passed value is not checked for validity.

   Algorithms:       None.
   Data Structures:  None.

   Global Variables: level - initialized to the passed value.

   Author:           Glen George
   Last Modified:    Mar. 8, 1994

*/

void  set_trg_level(int l)
{
    /* variables */
      /* none */



    /* set the trigger level */
    level = l;

    /* set the trigger level in hardware too */
    set_trigger(level, slope);


    /* all done initializing the trigger level - return */
    return;

}




/*
   trg_level_down

   Description:      This function handles decreasing the current trigger
   		     level.  The new trigger level is sent to the hardware.
		     If an attempt is made to lower the trigger level below
		     the minimum value it is not changed.

   Arguments:        None.
   Return Value:     None.

   Input:            None.
   Output:           None.

   Error Handling:   None.

   Algorithms:       None.
   Data Structures:  None.

   Global Variables: level - decremented if not already at the minimum value.

   Author:           Glen George
   Last Modified:    Mar. 8, 1994

*/

void  trg_level_down()
{
    /* variables */
      /* none */



    /* decrease the trigger level, if not already the minimum */
    if (level > MIN_TRG_LEVEL_SET)
        level--;

    /* set the trigger level for the hardware */
    set_trigger(level, slope);


    /* all done with lowering the trigger level - return */
    return;

}




/*
   trg_level_up

   Description:      This function handles increasing the current trigger
   		     level.  The new trigger level is sent to the hardware.
		     If an attempt is made to raise the trigger level above
		     the maximum value it is not changed.

   Arguments:        None.
   Return Value:     None.

   Input:            None.
   Output:           None.

   Error Handling:   None.

   Algorithms:       None.
   Data Structures:  None.

   Global Variables: level - incremented if not already the maximum value.

   Author:           Glen George
   Last Modified:    Mar. 8, 1994

*/

void  trg_level_up()
{
    /* variables */
      /* none */



    /* increase the trigger level, if not already the maximum */
    if (level < MAX_TRG_LEVEL_SET)
        level++;

    /* tell the hardware the new trigger level */
    set_trigger(level, slope);


    /* all done raising the trigger level - return */
    return;

}




/*
   display_trg_level

   Description:      This function displays the current trigger level at the
                     passed position, in the passed style.

   Arguments:        x_pos (int) - x position (in character cells) at which to
   				   display the trigger level.
   		     y_pos (int) - y position (in character cells) at which to
   				   display the trigger level.
		     style (int) - style with which to display the trigger
		     		   level.
   Return Value:     None.

   Input:            None.
   Output:           The trigger level is displayed at the passed position on
   		     the display.

   Error Handling:   None.

   Algorithms:       None.
   Data Structures:  None.

   Global Variables: level - determines the value displayed.

   Author:           Glen George
   Last Modified:    Mar. 10, 1995

*/

void  display_trg_level(int x_pos, int y_pos, int style)
{
    /* variables */
    char      level_str[] = "        "; /* string containing the trigger level */
    long int  l;			/* trigger level in mV */



    /* compute the trigger level in millivolts */
    l = ((long int) MAX_LEVEL - MIN_LEVEL) * level / (MAX_TRG_LEVEL_SET - MIN_TRG_LEVEL_SET) + MIN_LEVEL;

    /* convert the level to the string (leave first character blank) */
    cvt_num_field(l, &level_str[1]);

    /* add in the units */
    level_str[7] = 'V';


    /* now finally display the trigger level */
    plot_string(x_pos, y_pos, level_str, style);


    /* all done displaying the trigger level - return */
    return;

}




/*
   set_trg_slope

   Description:      This function sets the trigger slope to the passed value.

   Arguments:        s (enum slope_type) - trigger slope type to which to set
   					   the locally global slope.
   Return Value:     None.

   Input:            None.
   Output:           None.

   Error Handling:   None.

   Algorithms:       None.
   Data Structures:  None.

   Global Variables: slope - set to the passed value.

   Author:           Glen George
   Last Modified:    Mar. 8, 1994

*/

void  set_trg_slope(enum slope_type s)
{
    /* variables */
      /* none */



    /* set the slope type */
    slope = s;

    /* also tell the hardware what the slope is */
    set_trigger(level, slope);


    /* all done setting the trigger slope - return */
    return;

}




/*
   trg_slope_toggle

   Description:      This function handles toggling (and setting) the current
   		     trigger slope.

   Arguments:        None.
   Return Value:     None.

   Input:            None.
   Output:           None.

   Error Handling:   None.

   Algorithms:       None.
   Data Structures:  None.

   Global Variables: slope - toggled.

   Author:           Glen George
   Last Modified:    Mar. 8, 1994

*/

void  trg_slope_toggle()
{
    /* variables */
      /* none */



    /* toggle the trigger slope */
    if (slope == SLOPE_POSITIVE)
        slope = SLOPE_NEGATIVE;
    else
        slope = SLOPE_POSITIVE;

    /* set the new trigger slope */
    set_trigger(level, slope);


    /* all done with the trigger slope - return */
    return;

}




/*
   display_trg_slope

   Description:      This function displays the current trigger slope at the
                     passed position, in the passed style.

   Arguments:        x_pos (int) - x position (in character cells) at which to
   				   display the trigger slope.
   		     y_pos (int) - y position (in character cells) at which to
   				   display the trigger slope.
		     style (int) - style with which to display the trigger
		     		   slope.
   Return Value:     None.

   Input:            None.
   Output:           The trigger slope is displayed at the passed position on
   		     the screen.

   Error Handling:   None.

   Algorithms:       None.
   Data Structures:  None.

   Global Variables: slope - determines which string is displayed.

   Author:           Glen George
   Last Modified:    Mar. 13, 1994

*/

void  display_trg_slope(int x_pos, int y_pos, int style)
{
    /* variables */

    /* the trigger slope strings (must match enumerated type) */
    const char * const  slopes[] =  {  " +", " -"  };



    /* display the trigger slope */
    plot_string(x_pos, y_pos, slopes[slope], style);


    /* all done displaying the trigger slope - return */
    return;

}




/*
   set_trg_delay

   Description:      This function sets the trigger delay to the passed value.

   Arguments:        d (long int) - value to which to set the trigger delay.
   Return Value:     None.

   Input:            None.
   Output:           None.

   Error Handling:   The passed value is not checked for validity.

   Algorithms:       None.
   Data Structures:  None.

   Global Variables: delay - initialized to the passed value.

   Author:           Glen George
   Last Modified:    Mar. 8, 1994

*/

void  set_trg_delay(long int d)
{
    /* variables */
      /* none */



    /* set the trigger delay */
    delay = d;

    /* set the trigger delay in hardware too */
    set_delay(delay);


    /* all done initializing the trigger delay - return */
    return;

}




/*
   trg_delay_down

   Description:      This function handles decreasing the current trigger
   		     delay.  The new trigger delay is sent to the hardware.
		     If an attempt is made to lower the trigger delay below
		     the minimum value it is not changed.

   Arguments:        None.
   Return Value:     None.

   Input:            None.
   Output:           None.

   Error Handling:   None.

   Algorithms:       None.
   Data Structures:  None.

   Global Variables: delay - decremented if not already at the minimum value.

   Author:           Glen George
   Last Modified:    Mar. 8, 1994

*/

void  trg_delay_down()
{
    /* variables */
      /* none */



    /* decrease the trigger delay, if not already the minimum */
    if (delay > MIN_DELAY)
        delay--;

    /* set the trigger delay for the hardware */
    set_delay(delay);


    /* all done with lowering the trigger delay - return */
    return;

}




/*
   trg_delay_up

   Description:      This function handles increasing the current trigger
   		     delay.  The new trigger delay is sent to the hardware.
		     If an attempt is made to raise the trigger delay above
		     the maximum value it is not changed.

   Arguments:        None.
   Return Value:     None.

   Input:            None.
   Output:           None.

   Error Handling:   None.

   Algorithms:       None.
   Data Structures:  None.

   Global Variables: delay - incremented if not already the maximum value.

   Author:           Glen George
   Last Modified:    Mar. 8, 1994

*/

void  trg_delay_up()
{
    /* variables */
      /* none */



    /* increase the trigger delay, if not already the maximum */
    if (delay < MAX_DELAY)
        delay++;

    /* tell the hardware the new trigger delay */
    set_delay(delay);


    /* all done raising the trigger delay - return */
    return;

}




/*
   adjust_trg_delay

   Description:      This function adjusts the trigger delay for a new sweep
   		     rate.  The factor to adjust the delay by is determined
		     by looking up the sample rates in the sweep_rates array.
		     If the delay goes out of range, due to the adjustment it
		     is reset to the maximum or minimum valid value.

   Arguments:        old_sweep (int) - old sweep rate (index into sweep_rates
   				       array).
		     new_sweep (int) - new sweep rate (index into sweep_rates
   				       array).
   Return Value:     None.

   Input:            None.
   Output:           None.

   Error Handling:   None.

   Algorithms:       The delay is multiplied by 10 times the ratio of the
   		     sweep sample rates then divided by 10.  This is done to
		     avoid floating point arithmetic and integer truncation
		     problems.
   Data Structures:  None.

   Global Variables: delay - adjusted based on passed sweep rates.

   Known Bugs:       The updated delay time is not displayed.  Since the time
   		     is typically only rounded to the new sample rate, this is
		     not a major problem.

   Author:           Glen George
   Last Modified:    Mar. 8, 1994

*/

void  adjust_trg_delay(int old_sweep, int new_sweep)
{
    /* variables */
      /* none */



    /* multiply by 10 times the ratio of sweep rates */
    delay *= (10 * sweep_rates[new_sweep].sample_rate) / sweep_rates[old_sweep].sample_rate;
    /* now divide the factor of 10 back out */
    delay /= 10;

    /* make sure delay is not out of range */
    if (delay > MAX_DELAY)
        /* delay is too large - set to maximum */
        delay = MAX_DELAY;
    if (delay < MIN_DELAY)
        /* delay is too small - set to minimum */
	delay = MIN_DELAY;


    /* tell the hardware the new trigger delay */
    set_delay(delay);


    /* all done adjusting the trigger delay - return */
    return;

}




/*
   display_trg_delay

   Description:      This function displays the current trigger delay at the
                     passed position, in the passed style.

   Arguments:        x_pos (int) - x position (in character cells) at which to
   				   display the trigger delay.
   		     y_pos (int) - y position (in character cells) at which to
   				   display the trigger delay.
		     style (int) - style with which to display the trigger
		     		   delay.
   Return Value:     None.

   Input:            None.
   Output:           The trigger delay is displayed at the passed position on
   		     the display.

   Error Handling:   None.

   Algorithms:       None.
   Data Structures:  None.

   Global Variables: delay - determines the value displayed.

   Author:           Glen George
   Last Modified:    May 3, 2006

*/

void  display_trg_delay(int x_pos, int y_pos, int style)
{
    /* variables */
    char      delay_str[] = "         "; /* string containing the trigger delay */
    long int  units_adj;		 /* adjustment to get to microseconds */

    long int  d;                         /* delay in appropriate units */


    /* compute the delay in the appropriate units */
    /* have to watch out for overflow, so be careful */
    if (sweep_rates[sweep].sample_rate > 10000000L) {
    	d = delay * (1000000000L / sweep_rates[sweep].sample_rate);
    	/* need to divide by 1000000 to get milliseconds */
    	units_adj = 1000000;
    } else if (sweep_rates[sweep].sample_rate > 1000000L)  {
        /* have a fast sweep rate, could overflow */
        /* first compute in units of 100 ns */
        d = delay * (10000000L / sweep_rates[sweep].sample_rate);
		/* now convert to nanoseconds */
		d *= 100L;
		/* need to divide by 1000 to get to microseconds */
		units_adj = 1000;
    } else  {
        /* slow sweep rate, don't have to worry about overflow */
        d = delay * (1000000L / sweep_rates[sweep].sample_rate);
		/* already in microseconds, so adjustment is 1 */
		units_adj = 1;
    }

    /* convert it to the string (leave first character blank) */
    cvt_num_field(d, &delay_str[1]);

    /* add in the units */
    if (units_adj == 1000000) {
        /* delay is in nanoseconds */
		delay_str[7] = '\004';
		delay_str[8] = 's';
    } else if (((d / units_adj) < 1000) && ((d / units_adj) > -1000) && (units_adj == 1000)) {
        /* delay is in microseconds */
		delay_str[7] = '\004';
		delay_str[8] = 's';
    } else if (((d / units_adj) < 1000000) && ((d / units_adj) > -1000000)) {
        /* delay is in milliseconds */
		delay_str[7] = 'm';
		delay_str[8] = 's';
    } else if (((d / units_adj) < 1000000000) && ((d / units_adj) > -1000000000))  {
        /* delay is in seconds */
		delay_str[7] = 's';
		delay_str[8] = ' ';
    } else  {
        /* delay is in kiloseconds */
		delay_str[7] = 'k';
		delay_str[8] = 's';
    }


    /* now actually display the trigger delay */
    plot_string(x_pos, y_pos, delay_str, style);


    /* all done displaying the trigger delay - return */
    return;

}




/*
   cvt_num_field

   Description:      This function converts the passed number (numeric field
                     value) to a string and returns that in the passed string
		     reference.  The number may be signed, and a sign (+ or -)
		     is always generated.  The number is assumed to have three
		     digits to the right of the decimal point.  Only the four
		     most significant digits of the number are displayed and
		     the decimal point is shifted appropriately.  (Four digits
		     are always generated by the function).

   Arguments:        n (long int) - numeric field value to convert.
   		     s (char *)   - pointer to string in which to return the
		     		    converted field value.
   Return Value:     None.

   Input:            None.
   Output:           None.

   Error Handling:   None.

   Algorithms:       The algorithm used assumes four (4) digits are being
   		     converted.
   Data Structures:  None.

   Global Variables: None.

   Known Bugs:       If the passed long int is the largest negative long int,
   		     the function will display garbage.

   Author:           Glen George
   Last Modified:    Mar. 8, 1994

*/

void  cvt_num_field(long int n, char *s)
{
    /* variables */
    int  dp = 3;		/* digits to right of decimal point */
    int  d;			/* digit weight (power of 10) */

    int  i = 0;			/* string index */



    /* first get the sign (and make n positive for conversion) */
    if (n < 0)  {
        /* n is negative, set sign and convert to positive */
	s[i++] = '-';
	n = -n;
    }
    else  {
        /* n is positive, set sign only */
	s[i++] = '+';
    }


    /* make sure there are no more than 4 significant digits */
    while (n > 9999)  {
        /* have more than 4 digits - get rid of one */
	n /= 10;
	/* adjust the decimal point */
	dp--;
    }

    /* if decimal point is non-positive, make positive */
    /* (assume will take care of adjustment with output units in this case) */
    while (dp <= 0)
       dp += 3;


    /* adjust dp to be digits to the right of the decimal point */
    /* (assuming 4 digits) */
    dp = 4 - dp;


    /* finally, loop getting and converting digits */
    for (d = 1000; d > 0; d /= 10)  {

        /* check if need decimal the decimal point now */
	if (dp-- == 0)
	    /* time for decimal point */
	    s[i++] = '.';

	/* get and convert this digit */
	s[i++] = (n / d) + '0';
	/* remove this digit from n */
	n %= d;
    }


    /* all done converting the number, return */
    return;

}
\end{lstlisting}